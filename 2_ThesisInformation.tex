% This file requires some necessary information used in a writing.
% Some optional commands, mark with "OPT", can be omitted by making them as comments 
% (type "%" before them).

\pagestyle{topright}

% PART 1: About author and thesis info.
% ==============================================

% ชื่อวิจัยภาษาไทย - Research or Dissertation Title
% Research title for cover and other related pages 
\thesistitleThai{ชื่อเรื่องวิจัย} 
\thesistitle{Research or Dissertation Title}

% Research title for the abstract page
\thesistitleforabstractThai{ชื่อเรื่องวิจัย} % ใส่เหมือนกับข้างบน
\thesistitleforabstract{Research or Dissertation Title}

% Name and surename of a student
\authorThai{ชื่อผู้เขียน ชื่อสกุล}
\author{Firstname Surename}

% Name title of a student
\nametitle{คำนำหน้าชื่อ}

%% Date of birth of a student
%\dateofbirth{ชื่อเต็มของเดือน พ.ศ.}

%% Work position of a student while studying OPT
%\workposition{ชื่อตำแหน่งงานปัจจุบัน สังกัด} 

%% Educational attainment befor this study
%\eduattainment{ปีการศึกษา 25xx: ชื่อปริญญา\\
%	ชื่อมหาวิทยาลัย
%}

%% Scholarship OPT
%\scholarship{ปีพ.ศ.เรียงจากใหม่ไปหาเก่า: ชื่อทุนการศึกษาที่ได้รับ}

%% Work experiences OPT
%\workexperiences{ปีพ.ศ.เรียงจากใหม่ไปหาเก่า: ชื่อตำแหน่งงาน\\
%	สถานที่ทำงาน
%}

% Degree title of this study - Example: วิทยาศาสตรบัณฑิต
% ชื่อเต็มปริญญาที่ได้รับ เช่น ชื่อสาขาวิชา (วิทยาศาสตรบัณฑิต สาขาวิชาวิทยาการคอมพิวเตอร์และปัญญาประดิษฐ์)
\degreeThai{ระดับปริญญาตรี}
\degree{Bachelor's degree}

% ชื่อวิชาวิจัย - Research subject name - Example: วิธีวิจัยทางวิทยาการคอมพิวเตอร์และปัญญาประดิษฐ์
% ชื่อวิชาวิจัยภาษาไทย - การวิจัยทางวิทยาการคอมพิวเตอร์และปัญญาประดิษฐ์...
\researchsubjectThai{การวิจัยทางวิทยาการคอมพิวเตอร์และปัญญาประดิษฐ์}
\researchsubject{Name of Research subject in English}

% Major field of this study - วิทยาการคอมพิวเตอร์และปัญญาประดิษฐ์...
% ชื่อภาควิชาภาษาไทย - วิทยาการคอมพิวเตอร์และปัญญาประดิษฐ์...
\majorThai{วิทยาการคอมพิวเตอร์และปัญญาประดิษฐ์}
\major{Computer Science and Artificial Intelligence}

% Department - มหาวิทยาลัยราชภัฏจันทรเกษม ไม่ได้ใช้งานอันนี้ ไม่จำเป็นต้องเปลี่ยน
% \departmentThai{ภาควิชา}
% \department{department}

% Faculty - Example: คณะวิทยาศาสตร์
% ชื่อคณะภาษาไทย - คณะวิทยาศาสตร์ ...
\facultyThai{คณะวิทยาศาสตร์}
\faculty{Faculty in English}

% Academic year that this thesis is submitted - Example: 2568
% ปีการศึกษา เช่น 2568
\academicyear{2568}


% Fill a type of writing, e.g., Independent Study, Research or Dissertation.
% ประเภทของงานวิจัย
\typeofwritingThai{วิจัย}
\typeofwriting{Research}

% Approval date
\approvaldate{วันที่ ชื่อเต็มของเดือน พ.ศ. 2568}



% PART 2: Examination Committee with their academic degrees
% ==============================================

% Chairman - Unused
\chairman{ชื่อตำแหน่งทางวิชาการ ชื่อ ชื่อสกุลอาจารย์}
%\chairmandegree{Ph.D./M.D.}

% Head of Major
\majorhead{ชื่อตำแหน่งทางวิชาการ ชื่อ ชื่อสกุลอาจารย์}
%\majorheaddegree{Ph.D./M.D.} - Not update right now please

% Advisor - อาจารย์ที่ปรึกษา
\advisorThai{ชื่อตำแหน่งทางวิชาการ ชื่อ ชื่อสกุลอาจารย์}
\advisor{Academic Title Firstname Surname}
%\advisordegree{Ph.D./M.D.}

% Co-advisor OPT - อาจารย์ที่ร่วมปรึกษา
\coadvisorThai{ชื่อตำแหน่งทางวิชาการ ชื่อ ชื่อสกุลอาจารย์}
\coadvisor{Academic Title Firstname Surname}
%\coadvisordegree{Ph.D./M.D.}

% Other member
\memberone{ชื่อตำแหน่งทางวิชาการ ชื่อ ชื่อสกุลอาจารย์}
%\memberonedegree{Ph.D./M.D.}

\membertwo{ชื่อตำแหน่งทางวิชาการ ชื่อ ชื่อสกุลอาจารย์} %OPT
%\membertwodegree{Ph.D./M.D.} %OPT

% Dean of faculty - Unused
\dean{ชื่อตำแหน่งทางวิชาการ ชื่อ ชื่อสกุลอาจารย์}
%\deandegree{Ph.D./M.D.}




% PART 3: Abstract of this writing and keywords
% ==============================================

\begin{abstractThai}
	เนื้อหาบทคัดย่อ
\end{abstractThai}

\keywordsThai{พิมพ์คำสำคัญ, พิมพ์คำสำคัญ, พิมพ์คำสำคัญ}

\begin{abstract}
	Insert text here
\end{abstract}

\keywords{Insert keyword here, Insert keyword here, Insert keyword here}




% PART 4: Acknowledgements
% ==============================================
\begin{acknowledgements}
	เนื้อหากิตกรรมประกาศ
\end{acknowledgements}




% PART 5: Your Publications
% ==============================================
\begin{publications}
	ชื่อผลงานทางวิชาการ (ลงรายการอ้างอิง)
\end{publications}



% PART 6: List of Abbreviations
% ==============================================
% Symbols
\nomenclature[1ph]{$ \varphi $}{A Greek alphabet}
\nomenclature[1ps]{$ \psi $}{An other Greek alphabet}

% Alphabets
\nomenclature[2B]{$ \mathbf{B}(X,Y) $}{The set of all bounded linear operator from $ X $ to $ Y $ the set of all bounded linear operator from $ X $ to $ Y $}
\nomenclature[2R]{$ \mathbb{R} $}{The set of real numbers}
\nomenclature[2R]{R}{The 18th of English alphabets}




