% BIBAUTO
\makeatletter
\ifdefined\@bibauto
\ifx\@bibauto\bibvancouver
% biblatex vancouver
\printbibliography[title=\bibname]
\else
% biblatex apa, biblatex chicago-notes (Turabian)
\printbibheading[title=\bibname]
\nobibintoc
\printbibliography[heading=subbibliography, keyword=bookth, title={หนังสือและบทความในหนังสือ}] % thai references
% use heading=none to remove heading of the list of references whose keyword is book and combine them with the list of references above.
\printbibliography[heading=none, keyword=book, title={หนังสือและบทความในหนังสือ}] % english references
\printbibliography[heading=subbibliography, keyword=proceeding, title={รายงานการประชุมทางวิชาการ}]
\printbibliography[heading=subbibliography, keyword=article, title={บทความวารสาร}]
\printbibliography[heading=subbibliography, keyword=thesis, title={วิทยานิพนธ์}]
\printbibliography[heading=subbibliography, keyword=report, title={รายงาน}]
\printbibliography[heading=subbibliography, keyword=manual, title={คู่มือ}]
\printbibliography[heading=subbibliography, keyword=booklet, title={จุลสาร}]
\printbibliography[heading=subbibliography, keyword=electronicmedia, title={สื่ออิเล็กทรอนิกส์}]
\printbibliography[heading=subbibliography, keyword=unpublished, title={เอกสารที่ไม่ตีพิมพ์}]
\printbibliography[heading=subbibliography, keyword=miscellaneous, title={เบ็ดเตล็ด}]
\fi\fi\makeatother





% BIBMANUAL
\makeatletter\ifdefined\@bibmanual
\begin{thebibliography}{}
	
	\item\textbf{หนังสือและบทความในหนังสือ} % remove when the vancouver style is active.
	\item % remove when the vancouver style is active.
	
	
	% set up references for turabian and apa styles
	\bibitem[First Author et al.(year)First Author, Second Author, and Third Author]{ID1}
	First Author, Second Author, and Third Author...
	
	
	\bibitem[First Author and Second Author(year)]{ID2}
	First Author, Second Author, and Third Author...
	
	
	\item % remove when the vancouver style is active.
	\item\textbf{บทความวารสาร} % remove when the vancouver style is active.
	\item % remove when the vancouver style is active.
	
	
	% example
	\bibitem[Jones et al.(1990)Jones, Baker, and Williams]{jon90}
	Jones, Baker, and Williams...
	
	\bibitem[James and Tony(1991)]{jam91}
	James and Tony...
	
	\bibitem[Jones et al.(1991)Jones, Robin, and Smith]{jon91}
	Jones, Robin, and Smith...
	
	
	\item % remove when the vancouver style is active.
	\item\textbf{วิทยานิพนธ์} % remove when the vancouver style is active.
	\item % remove when the vancouver style is active.
	
	
	\bibitem[Jones et al.(1990a)Jones, Baker, and Williams]{jon90a}
	Jones, Baker, and Williams...
	
	\bibitem[Jones et al.(1990b)Jones, Robin, and Smith]{jon90b}
	Jones, Robin, and Smith...
	
	\bibitem[Jones et al.(1990a)Jones, Baker, and Williams]{jon90aa}
	Jones, Baker, and Williams...
	
	\bibitem[Jones et al.(1990b)Jones, Robin, and Smith]{jon90ba}
	Jones, Robin, and Smith...
	
	\bibitem[Jones et al.(1990a)Jones, Baker, and Williams]{jon90ab}
	Jones, Baker, and Williams...
	
	\bibitem[Jones et al.(1990b)Jones, Robin, and Smith]{jon90bb}
	Jones, Robin, and Smith...
	
	
	
\end{thebibliography}
\fi\makeatother











